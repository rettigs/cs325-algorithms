\documentclass{article}
\usepackage[utf8]{inputenc}
\usepackage{mathtools}
\usepackage{amssymb}

\usepackage{geometry}
\geometry{letterpaper, portrait, margin=1in}

\title{CS325 - Project 1}
\author{Alexander Merrill}
\date{October 2014}

\begin{document}

\maketitle

\section*{Proof of Claim 1}

\begin{quote}
    Claim 1: $y_i$ is not visible iff $\exists j,k$ such that $j < i < k$ and $y* > m_i x* + b_i$ where $(x*,y*)$ is the intersection of $y_j$ and $j_k$.
\end{quote}\\
\\
$A \equiv y_i$ is not visible\\
$B \equiv \exists j,k$ such that $j < i < k$ and $y* < m_i x* + b_i$ where $(x*,y*)$ is the intersection of $y_j$ and $y_k$.\\
$A \Leftrightarrow B$

\subsection*{First Prove $A \Rightarrow B$}
    \subsubsection*{Direct Proof:}
    Let $y_i$ be a line that is not visible.\\
    Then $l < i < n$ because $y_i$ and $y_n$ are always visible.\\
    Let $k$ be the smallest index greater than $i$ such that $y_k$ is visible.\\
    e.g. $y_1,y_2,...,y_k,y_{k+1},...,y_{n-1},y_n$\\
    Let $(x*,y*)$ be the left most point on $y_k$ that is visible.\\
    Let $j$ be the greatest index such that $y_i$ intersects $y_k$ at $(x*,y*)$ is visible.\\
    Because $y_i$ through $y_{k-1}$ are not visible (by definition of $k_j$) $j < i < k$.\\
    Since $x*,y*$ is visible and $y_i$ is not visible, $m_i x + b_i < y*$.
    
\subsection*{Prove $B \Rightarrow A$}
    \subsubsection*{Direct Proof:}
    Since $m_i < m_k$, the intersection point of $y_i$ and $y_k$ is left of $x*$.\\
    Since $m_i < m_k$, $m_i x + b_i < m_k x + b_k$ $\forall{x > \bar{x}}$.\\
    Likewise since $m_i > m_j$, $y_i$ and $y_j$ intersect at $(\bar{\bar{x}}, \bar{\bar{y}})$ right of $x*$ $(\bar{\bar{x}} > x*)$.\\
    $\therefore m_i x + b_i < m_j x + b_j$; $\forall{x < \bar{\bar{x}}}$.\\
    $\therefore y_i$ is not visible.\\
    $y_k + y_j$ intersect at $m_k x + b_k = m_j x + b_j$\\
    $x = \dfrac{(b_j - b_k)}{(m_k - m_j)}$\\
    Is $m_j \left( \dfrac{b_j - b_k}{m_k - m_j}\right) + b_j > m_i \left(\dfrac{b_j - b_k}{m_k - m_j}\right) + b_i$\\
    If $m_k > m_j$ then instead compare $m_j (b_j - b_k) + b_j (m_k - m_j) > m_i (b_j + b_k)$

\section*{Proof of Claim 2}
\begin{quote}
    Claim 2: If $\{y_1, y_2,...,y_{j_{t}}\}$ is the visible subset of $\{y_1, y_2,...,y_{i - 1}\} (t \leq i - 1)$ then $\{y_1, y_2,...,y_{j_{k}}, y_i\}$ is the visible subset of $\{y_1, y_2,...,y_{i}\}$ where $y_{j_{k}}$ is the last line such that $y_{j_{k}} (x*) > y_i (x*)$ where $(x*, y_{j_{k}}(x*))$ is the point of intersection of lines $y_{j_{k}}$ and $y_{j_{k - 1}}$.
\end{quote}\\
\\
\subsubsection*{Proof by Inductions:}
Known:\\
The lines with the greatest and least slope magnitudes are always visible.\\
The array is sorted from least to greatest slope magnitude.\\
Base Case:\\
If $i \leq 2$ then all lines are visible.\\
Inductive Hypothesis:\\
Let $\{y_1, y_2,...,y_{j_{m}}\}$ be the visible subset of $\{y_1, y_2,...,y_{o - 1}\} (t \leq o - 1)$ then $\{y_1, y_2,...,y_{j_{n}}, y_i\}$ is the visible subset of $\{y_1, y_2,...,y_{i}\}$ where $y_{j_{n}}$ is the last line such that $y_{j_{n}} (x*) > y_o (x*)$ where $(x*, y_{j_{n}}(x*))$ is the point of intersection of lines $y_{j_{n}}$ and $y_{j_{n - 1}}$.\\
\\
If $y_o (x*) > y_o (x*)$ then 
$y_{j_n}(x*) < y_o (x*)$ then 

Assume that any array of length n < i is correctly flagged visible.
Apply the Axiom of Induction:\\
A is an array of length o.\\

If $y_o (x*) > y_o (x*)$ then \\
If $y_{j_n}(x*) < y_{o + 1} (x*)$ then remove $y_{j_n}(x*)$ from the visibility array and recurse.\\
If $y_{j_n}(x*) \geq y_{o + 1} (x*)$\\
\\
If $i = 3$ then let A_v $\{y_1, y_2\}$ be the visible subset of set A $\{y_1, y_2\}$.\\
Let the new set B be $\{y_1, y_2, y_3\}$.\\
Let the new initial visible set of B, $B_{v_{0}}$, be $A_v + y_3$, that is $\{y_1, y_2, y_3\}$.\\
It's initial visible subset will be A_v.\\
Let $(x*, y_2 (x*))$ be the point of intersection of $y_2$ and $y_{2 - 1} = y_1$.\\
Then let $y_{2 + 1} (x*) = y_3 (x*)$.\\
Then let $y_3$ be y coordinate at point $(x*, y_3 (x*))$, above the intersection of $y_2$ and $y_{2 - 1} = y_1$.\\
If $y_3 (x*) > y_2 (x*)$ then remove $y_2$ from $B_{v_0}$ and \\
Otherwise add $y_3$ to $B_{v_0}$\\


\section*{Group Proof of Claim 2}
\begin{quote}
    Claim 2: If $\{y_1, y_2,...,y_{j_{t}}\}$ is the visible subset of $\{y_1, y_2,...,y_{i - 1}\} (t \leq i - 1)$ then $\{y_1, y_2,...,y_{j_{k}}, y_i\}$ is the visible subset of $\{y_1, y_2,...,y_{i}\}$ where $y_{j_{k}}$ is the last line such that $y_{j_{k}} (x*) > y_i (x*)$ where $(x*, y_{j_{k}}(x*))$ is the point of intersection of lines $y_{j_{k}}$ and $y_{j_{k - 1}}$.
\end{quote}\\
\\
\subsubsection*{Proof by Induction:}
Let $A$ be $\{y_{1}, y_2, ..., y_{i - 1}\}$ and $A^+$ be $\{y_1, y_2, ..., y_{i}\}$.\\
Let $V$ be $\{y_{j_1}, y_{j_2}, ..., y_{j_t}\}$ and $V^+$ be $\{y_{j_1}, y_{j_2}, ..., y_{j_k}, y_{i}\}$.\\
\subsubsection*{Base Case:}
If $size(A) < 2$, $V^+$ trivially contains $V$ and $y_i$ because none of the line's slopes are the same and one or two lines with different slopes cannot cover one another.\\
\subsubsection*{Inductive Hypothesis:}
The claim is true when A (and V) have at least two lines.\\
\subsubsection*{Apply the Axiom of Induction:}
Pick off smallest with $size(A) = size(V) \geq 2$.\\
\subsubsubsection{There are two possibilities:}
\begin{enumerate}
    \item $y_i$ does not cover a line in $V$\\
    So, $V = \{y_1, ..., y_{z-1}, y_z\}$ and $y_z(x^*) \geq y_i$ where $x^*$ is the intersection of $y_{z-1}$ and $y_z$\\
    So, $y_z$ is not covered. $V^+ = V \cup y_i$.

    \item $y_i$ covers a line in $V$\\
    So $V = \{y_1, ..., y_{z-1}, y_z\}$ and $y_z(x*) < y_i(x*)$\\
    So $y_z$ is covered.\\
    To determine visibility, recurse with same $y_i$ and $A$ but remove $y_z$ from $V$.\\
\end{enumerate}

\end{document}
